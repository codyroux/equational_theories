\chapter{Rewriting theory}\label{rewriting-chapter}

We briefly recall the basics of rewrite theory necessary to our exposition, following mostly Baader and Nipkow \cite{traat}, and generally omitting proofs when they can be found there.

We first work in the abstract taking an arbitrary set $A$, with a given equivalence relation over it which we denote $\approx$. We consider a relation $R$ over $A$.
\begin{definition}
  We write $a \rightarrow b$ if $a\ R\ b$ holds in $A$, and say that $a$ \emph{rewrites to} (or \emph{reduces to}) $b$. We further define
  \begin{itemize}
  \item $\rightarrow^+$ as the transitive closure of $R$.
  \item $\rightarrow^*$ as the reflexive transitive closure of $R$.
  \item $\leftrightarrow^*$ as the reflexive transitive and symmetric closure of $R$
  \end{itemize}
  
  we sometimes write $b\leftarrow a$, (resp. $b{}^*\leftarrow a$ etc) to mean $a\rightarrow b$ (resp. $a\rightarrow^*b$ ect), and chain notations, e.g. $b_1\leftarrow a\rightarrow b_2$.
\end{definition}

Note that $\leftrightarrow^*$ is an equivalence relation and the hope is for it to be equal to $\approx$, in order to deduce properties of the latter.

One should first note that if even $R$ is contained in $\approx$, then so are $\rightarrow^+, \rightarrow^*$ and $\leftrightarrow^*$ (as it is an equivalence relation), so we will focus on that case. Generally $a \rightarrow^* b$ can be seen as a way to \emph{compute} the $\approx$ relation, as it is directed, in a way to constrain our search space.

However in general, we cannot deduce the converse, so it may be the case that $a\approx b$ but neither $a\rightarrow^*b$ nor $b\rightarrow^*a$ nor even is there a single $c$ such that $a\rightarrow^* c{}^*\leftarrow b$, as the number of ``left-right alternations'' may be arbitrarily large.

The following properties are going to be very useful to deduce exactly such a converse.

\begin{definition}
  We say that $R$ is \emph{Church-Rosser} if whever $a\leftrightarrow^* b$, there exists some $c$ such that
  \[a\rightarrow^* c{}^*\leftarrow b\]
  We say that $R$ is \emph{confluent} if whenever $b_1{}^*\leftarrow a\rightarrow^* b_2$ there exists some $c$ such that $b_1\rightarrow^* c{}^*\leftarrow b_2$.
  
  We say that $R$ is \emph{weakly confluent} if whenever $b_1\leftarrow a\rightarrow b_2$ there exists some $c$ such that $b_1\rightarrow^* c{}^*\leftarrow b_2$.
  
  We say that (an arbitrary) $a$ is in \emph{normal form} (or $a$ is a normal form) if there is no $a' \neq a$ such that $a \rightarrow a'$, and that $R$ is \emph{weakly normalizing} if for every $a$, there is some $a'$ such that $a\rightarrow^*a'$ and $a'$ is in normal form.

  We say that $R$ is \emph{strongly normalizing} if there are no infinite rewrite sequences $a_1\rightarrow a_2\rightarrow \ldots$. In particular, a strongly normalizing $R$ is also weakly normalizing.
\end{definition}

It turns out that if $R$ is strongly normalizing and Church-Rosser, and effective (we can ``compute'' with it) then the problem of equivalence is decidable! This is because of the following lemma.

\begin{lemma}
  If $R$ is Church-Rosser, then any normal form is \emph{unique}.
\end{lemma}

This means that, in this situation, $a$ and $b$ reduce to an identical normal form $c$ \emph{if and only if} $a\leftrightarrow^* b$! This means that we have the following algorithm to decide $a\leftrightarrow^* b$ (and therefore $a\approx b$ if these relations coincide):
\begin{enumerate}
\item Repeatedly apply $R$ to $a$ and $b$ until normal forms $a'$ and $b'$ are found for them (this is possible because $R$ is strongly normalizing).
\item Compare $a'$ and $b'$ for exact equality (sometimes called ``syntactic equality'').
\item If $a' = b'$, we can conclude $a\leftrightarrow^* b$
\item If $a' \neq b'$ can can conclude that they are \emph{not} equivalent due to the lemma.
\end{enumerate}

Note that weak normalization doe not change much here except at step 1, where we need to pick reductions which eventually bring the elements to normal forms.

The strategy is therefore, for a given $\approx$ to find an $R$ which is (strongly) normalizing and Church-Rosser, and such that $\leftrightarrow^* = \approx$. This is roughly the goal of the entire field of \emph{completion}.

The task is helped by the following facts, which we state here also without proof.

\begin{theorem}
  \begin{itemize}
  \item $R$ is Church-Rosser iff it is confluent.
  \item (Newmann's lemma) if $R$ is strongly normalizing, then $R$ is confluent iff it is weakly confluent.
  \end{itemize}
\end{theorem}

This can be leveraged by looking at the particulars of the equivalence relation of interest, namely quantified equations over syntactic trees.

%% TODO: Explain orienting equations, congruence, and critical pairs (with the critical pair lemma).
