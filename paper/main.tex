\documentclass[12pt]{amsart}

% Packages
\usepackage{amsmath, amssymb, amsthm}
\usepackage{geometry}
\usepackage{hyperref}
\usepackage{cleveref}
\usepackage{graphicx}
\usepackage{enumitem}
\usepackage{color}
\usepackage{mathtools}
\usepackage{tikz}
\usepackage{tikz-cd}
\usepackage{quiver}
\usepackage{mathrsfs}
\usepackage{proof}
\usepackage{todonotes}
\usepackage{siunitx}
\usepackage[utf8]{inputenc}
\usepackage[T1]{fontenc}
\usepackage{fancyvrb}
\fvset{commandchars=\\\{\}}
\usepackage{pmboxdraw}
\usetikzlibrary{positioning,arrows.meta,fit}

\tikzset{
  human/.style   = {->, very thick, color=black, shorten >=2pt},
  auto/.style    = {->, thin, dashed, color=black!60, shorten >=2pt},
  semi/.style    = {->, thick, dash pattern=on 4pt off 2pt on 1pt off 2pt, color=black!80, shorten >=2pt}
}


% Page Setup
\geometry{letterpaper, margin=1in}
\setlength{\parindent}{0pt} % No indent for paragraphs
\setlength{\parskip}{1em}   % Spacing between paragraphs

% Theorem Styles
\newtheorem{theorem}{Theorem}[section]
\newtheorem{lemma}[theorem]{Lemma}
\newtheorem{proposition}[theorem]{Proposition}
\newtheorem{corollary}[theorem]{Corollary}
\theoremstyle{definition}
\newtheorem{definition}[theorem]{Definition}
\newtheorem{example}[theorem]{Example}
\newtheorem{remark}[theorem]{Remark}

% Commands
\newcommand{\R}{\mathbb{R}}
\newcommand{\C}{\mathbb{C}}
\newcommand{\N}{\mathbb{N}}
\newcommand{\Z}{\mathbb{Z}}
\newcommand{\Q}{\mathbb{Q}}
\newcommand{\F}{\mathbb{F}}
\newcommand{\x}{\mathrm{x}}
\newcommand{\y}{\mathrm{y}}
\newcommand{\z}{\mathrm{z}}
\newcommand{\w}{\mathrm{w}}
\newcommand{\uu}{\mathrm{u}}
\newcommand{\vv}{\mathrm{v}}
\newcommand{\op}{\diamond}
\newcommand{\formaleq}{\simeq}
\newcommand{\eps}{\varepsilon}
\newcommand{\vdashfin}{\vdash_{\mathrm{fin}}}
\newcommand{\nvdashfin}{\nvdash_{\mathrm{fin}}}
\newcommand{\note}[1]{{\bf #1}}
\newcommand{\Magma}{{\mathcal{M}}}
\newcommand{\MagmaN}{{\mathcal{N}}}
\newcommand{\Eq}[1]{\mathrm{E}#1}
\newcommand{\TODO}[1]{\todo[inline]{#1}}


% Title Information
\title[Equational Theories Project]{The Equational Theories Project: Advancing Collaborative Mathematical Research at Scale}
\author[Equational Theories Project contributors]{Matthew Bolan, Joachim Breitner, Jose Brox, Mario Carneiro,
  Martin Dvorak, Andr\'es Goens, Aaron Hill, Harald Husum, Zoltan Kocsis, Bruno Le Floch, Lorenzo Luccioli, Douglas McNeil,
  Alex Meiburg, Pietro Monticone, Giovanni Paolini, Marco Petracci, Bernhard Reinke, David Renshaw, Marcus Rossel, Cody Roux,
  J\'er\'emy Scanvic, Shreyas Srinivas, Anand Rao Tadipatri, Terence Tao, Vlad Tsyrklevich,
  Daniel Weber, Fan Zheng}
\date{\today}

\begin{document}

\begin{abstract}
  We report on the \emph{Equational Theories Project} (ETP), an online collaborative pilot project
  to explore new ways to collaborate in mathematics with machine assistance. The project successfully determined all $\num{22028942}$ edges of the implication graph between the $4694$ simplest equational laws on magmas, by a combination of
  human-generated and automated proofs, all validated by the formal proof assistant language
  \emph{Lean}. As a result of this project, several new constructions of magmas obeying specific laws were discovered, and several auxiliary questions were also addressed, such as the effect of restricting attention to finite magmas.
\end{abstract}

\maketitle

\tableofcontents

\chapter{Basic theory of magmas}

\begin{definition}[Magma]\label{magma-def}\lean{Magma}\leanok A \emph{magma} is a set $G$ equipped with a binary operation $\circ: G \times G \to G$.  A \emph{homomorphism} $\varphi : G \to H$ between two magmas is a map such that $\varphi(x \circ y) = \varphi(x) \circ \varphi(y)$ for all $x,y \in G$.  An \emph{isomorphism} is an invertible homomorphism.
\end{definition}

Groups, semi-groups, and monoids are familiar examples of magmas.  However, in general we do not expect magmas to have any associative properties.  In some literature, magmas are also known as groupoids, although this term is also used for a slightly different object (a category with inverses).

A magma is called \emph{empty} if it has cardinality zero, \emph{singleton} if it has cardinality one, and \emph{non-trivial} otherwise.

The number of magma structures on a set $G$ of cardinality $n$ is of course $n^{n^2}$, which is \footnote{All sequences start from $n=0$ unless otherwise specified.}
$$ 1, 1, 16, 19683, 4294967296, 298023223876953125, \dots$$
(\href{https://oeis.org/A002489}{OEIS A002489}).
Up to isomorphism, the number of finite magmas of cardinality $n$ up to isomorphism is the slightly slower growing sequence
$$ 1, 1, 10, 3330, 178981952, 2483527537094825, 14325590003318891522275680, \dots$$
(\href{https://oeis.org/A001329}{OEIS A001329}).

\begin{definition}[Free Magma]\label{free-magma-def}\lean{FreeMagma}\leanok\uses{magma-def} The \emph{free magma} $M_X$ generated by a set $X$ (which we call an \emph{alphabet}) is the set of all finite formal expressions built from elements of $X$ and the operation $\circ$.  An element of $M_X$ will be called a \emph{word} with alphabet $X$.  The \emph{order} of a word is the number of $\circ$ symbols needed to generate the word.  Thus for instance $X$ is precisely the set of words of order $0$ in $M_X$.
\end{definition}

For sake of concreteness, we will take the alphabet $X$ to default to the natural numbers $\N$ if not otherwise specified.

For instance, if $X = \{0,1\}$, then $M_X$ would consist of the following words:
\begin{itemize}
  \item $0$, $1$ (the words of order $0$);
  \item $0 \circ 0$, $0 \circ 1$, $1 \circ 0$, $1 \circ 1$ (the words of order $1$);
  \item $0 \circ (0 \circ 0)$, $0 \circ (0 \circ 1)$, $0 \circ (1 \circ 0)$, $0 \circ (1 \circ 1)$, $1 \circ (0 \circ 0)$, $1 \circ (0 \circ 1)$, $1 \circ (1 \circ 0)$, $1 \circ (1 \circ 1)$, $(0 \circ 0) \circ 0$, $(0 \circ 0) \circ 1$, $(0 \circ 1) \circ 0$, $(0 \circ 1) \circ 1$, $(1 \circ 0) \circ 0$, $(1 \circ 0) \circ 1$, $(1 \circ 1) \circ 0$, $(1 \circ 1) \circ 1$ (the words of order $2$);
  \item etc.
\end{itemize}

\begin{lemma} \leanok \lean{FreeMagma.elementsOfNumNodesEq_card_eq_catalan_mul_pow} For a finite alphabet $X$, the number of words of order $n$ is $C_n |X|^{n+1}$, where $C_n$ is the $n^{\mathrm{th}}$ Catalan number and $X$ is the cardinality of $X$.
\end{lemma}

\begin{proof} \leanok Follows from standard properties of Catalan numbers.
\end{proof}

The first few Catalan numbers are
$$ 1, 1, 2, 5, 14, 42, 132, \dots$$
(\href{https://oeis.org/A000108}{OEIS A000108}).


\begin{definition}[Induced homomorphism]\label{induced-def}\uses{free-magma-def}  Given a function $f: X \to G$ from an alphabet $X$ to a magma $G$, the \emph{induced homomorphism} $\varphi_f: M_X \to G$ is the unique extension of $f$ to a magma homomorphism.  Similarly, if $\pi \colon X \to Y$ is a function, we write $\pi_* \colon M_X \to M_Y$ for the unique extension of $\pi$ to a magma homomorphism.
\end{definition}

For instance, if $f : \{0,1\} \to G$ maps $0,1$ to $x,y$ respectively, then
$$ \varphi_f(0 \circ 1) = x \circ y$$
$$ \varphi_f(1 \circ (0 \circ 1)) = y \circ (x \circ y)$$
and so forth.  If $\pi \colon \N \to \N$ is the map $\pi(n) := n+1$, then
$$ \pi_*(0 \circ 1) = 1 \circ 2$$
$$ \pi_*(1 \circ (0 \circ 1)) = 2 \circ (1 \circ 2)$$
and so forth.

\begin{definition}[Law]\label{law-def}
  \lean{Law.MagmaLaw}\leanok
  \uses{induced-def}
  Let $X$ be a set. A \emph{law} with alphabet $X$ is a formal expression of the form $w \formaleq w'$,
  where $w, w' \in M_X$ are words with alphabet $X$ (thus one can identify laws with alphabet $X$
  with elements of $M_X \times M_X$).  A magma $G$ \emph{satisfies} the law $w \formaleq w'$ if
  we have $\varphi_f( w ) = \varphi_f ( w' )$ for all $f: X \to G$, in which case we write
  $G \models w \formaleq w'$.
\end{definition}

Thus, for instance, the commutative law
\begin{equation}\label{comm-law}
  0 \circ 1 \formaleq 1 \circ 0
\end{equation}
is satisfied by a magma $G$ if and only if
\begin{equation}\label{comm-law-2}
 x \circ y = y \circ x
\end{equation}
for all $x, y \in G$.  We refer to \eqref{comm-law-2} as the \emph{equation} associated to the law \eqref{comm-law}.  One can think of equations as the ``semantic'' interpretation of a ``syntactic'' law.  However, we shall often abuse notation and a law with its associated equation thus we shall (somewhat carelessly) also refer to \eqref{comm-law-2} as ``the commutative law'' (rather than ``the commutative equation'').

\begin{definition}[Models]\label{models-def}
  \lean{models}\leanok
  \uses{law-def}
  Given an arbitrary set $\Gamma$ of laws, a magma $G$ is a \emph{model} of $\Gamma$ with the
  (overloaded) notation $G\models\Gamma$ if $G\models w\formaleq w'$ for every $w\formaleq w'$ in $\Gamma$; we also say that $G$ \emph{obeys} $\Gamma$.  Given a law $E$, we write $\Gamma \models E$ if every magma $G$ that models $\Gamma$, also models $E$.
\end{definition}

\begin{definition}[Derivation]\label{derivation-def}
  \lean{derive}\leanok
  \uses{law-def}
  Given a set $\Gamma$ of laws and a law $w\formaleq w'$ over a fixed alphabet $X$, we say that
  $\Gamma$ \emph{derives} $w\formaleq w'$, and write $\Gamma\vdash w\formaleq w'$, if the law can
  be obtained using a finite number of applications of the following rules:
  \begin{enumerate}
    \item if $w\formaleq w' \in \Gamma$, then $\Gamma\vdash w\formaleq w'$.
    \item $\Gamma\vdash w\formaleq w$ for any word $w$.
    \item if $\Gamma\vdash w\formaleq w'$ then $\Gamma\vdash w'\formaleq w$.
    \item if $\Gamma\vdash w\formaleq w'$ and $\Gamma\vdash w'\formaleq w''$ then $\Gamma\vdash w\formaleq w''$.
    \item if $\Gamma\vdash w\formaleq w'$ then $\Gamma\vdash w\sigma\formaleq w'\sigma$, where $\sigma$ is an arbitrary map from $X$ to words in $M_X$ and $w\sigma$ replaces each occurrence of an element of $X$ with it's image by $\sigma$ in $w$.
    \item if $\Gamma\vdash w_1\formaleq w_2$ and $\Gamma\vdash w_3\formaleq w_4$ then $\Gamma\vdash w_1 \circ w_3\formaleq w2\circ w_4$
  \end{enumerate}
\end{definition}

This definition is useful because of the following theorem:

\begin{theorem}[Birkhoff's completeness theorem]\label{sound-complete}\lean{Completeness}\leanok\uses{models-def, derivation-def}
  For any set of laws $\Gamma$ and words $w, w'$ over a fixed alphabet
  $$ \Gamma\vdash w\formaleq w'\ \mathrm{iff}\ \Gamma\models w\formaleq w'.$$
\end{theorem}
\begin{proof}
  \leanok
  (Sketch) The `only if' component is soundness, and follows from verifying that the rules of inference in Definition \ref{derivation-def} holds for $\models$.  The `if` part is completeness, and is proven by constructing the magma of words, quotiented out by the relation $\Gamma \vdash w \formaleq w'$, which is easily seen to be an equivalence relation respecting the magma operation.
\end{proof}


\begin{corollary}[Compactness theorem]\label{compactness-thm}\uses{models-def}  Let $\Gamma$ be a collection of laws, and let $E$ be a law.  Then $\Gamma \models E$ if and only if there exists a finite subset $\Gamma'$ of $\Gamma$ such that $\Gamma' \models E$.
\end{corollary}

\begin{proof}\uses{sound-complete} The claim is obvious for $\vdash$, and the claim then follows from Theorem \ref{sound-complete}.
\end{proof}


\begin{lemma}[Pushforward]\label{push}\uses{law-def}  Let $w \formaleq w'$ be a law with some alphabet $X$, $G$ be a magma, and $\pi: X \to Y$ be a function.  If $G \models w \formaleq w'$, then $G \models \pi_*(w) \formaleq \pi_*(w')$.  In particular, if $\pi$ is a bijection, the statements If $G \models w \formaleq w'$, then $G \models \pi_*(w) \formaleq \pi_*(w')$ are equivalent.
\end{lemma}

\begin{proof}  Trivial.
\end{proof}

If $\pi$ is a bijection, we will call $\pi_*(w) \formaleq \pi_*(w')$ a \emph{relabeling} of the law $w \formaleq w'$.  Thus for instance
$$ 5 \circ 7 \formaleq 7 \circ 5$$
is a relabeling of the commutative law \eqref{comm-law}.  By the above lemma, relabeling does not affect whether a given magna satisfies a given law.

\begin{lemma}[Equivalence]\label{equiv}\uses{law-def}  Let $G$ be a magma and $X$ be an alphabet.  Then the relation $G \models w \formaleq w'$ is an equivalence relation on $M_X$.
\end{lemma}

\begin{proof}  Trivial.
\end{proof}

Define the total order of a law $w \formaleq w'$ to be the sum of the orders of $w$ and $w'$.

\begin{lemma}[Counting laws up to relabeling]\label{law-count}\uses{push}  Up to relabeling, the number of laws $w \formaleq w'$ of total order $n$ is $C_{n+1} B_{n+2}$.
\end{lemma}

\begin{proof} Follows from the properties of Catalan and Bell numbers.
\end{proof}

The first few Bell numbers are
$$ 1, 1, 2, 5, 15, 52, 203, \dots$$
(\href{https://oeis.org/A000110}{OEIS A000110}).

The sequence in Lemma \ref{law-count} is
$$ 2, 10, 75, 728, 8526, 115764, \dots$$
(\href{https://oeis.org/A289679}{OEIS A289679}).

Now we would also like to count laws up to relabeling and symmetry.

\begin{lemma}[Counting laws up to relabeling and symmetry]\label{law-count-sym}\uses{push} Up to relabeling and symmetry, the number of laws $w \formaleq w'$ of total order $n$ is
$$ C_{n+1} B_{n+2}/2$$
when $n$ is odd, and
$$ (C_{n+1} B_{n+2} + C_{n/2} (2D_{n+2} - B_{n+2}))/2$$
when $n$ is even, where $D_n$ is the number of partitions of $[n]$ up to reflection.
\end{lemma}

\begin{proof} Elementary counting.
\end{proof}

The sequence $D_n$ is
$$ 1, 1, 2, 4, 11, 32, 117, \dots$$
(\href{https://oeis.org/A103293}{OEIS A103293}), and the sequence in Lemma \ref{law-count-sym} is
$$ 2, 5, 41, 364, 4294, 57882, 888440, \dots$$
(\href{https://oeis.org/A376620}{OEIS A376620}).

We can also identify all laws of the form $w \formaleq w$ with the trivial law $0 \formaleq 0$.  The number of such laws of total order $n$ is zero if $n$ is odd, and $C_{n/2} B_{n/2+1}$ if $n$ is even.  We conclude:

\begin{lemma}[Counting laws up to relabeling, symmetry, and triviality]  Up to relabeling, symmetry, and triviality, the number of laws of total order $n$ is
$$ C_{n+1} B_{n+2}/2$$
if $n$ is odd, $2$ if $n = 0$, and
$$ (C_{n+1} B_{n+2} + C_{n/2} (2D_{n+2} - B_{n+2}))/2 - C_{n/2} B_{n/2+1}$$
if $n \geq 2$ is even.
\end{lemma}

\begin{proof} Routine counting.
\end{proof}

This sequence is
$$2, 5, 39, 364, 4284, 57882, 888365, \dots$$
(\href{https://oeis.org/A376640}{OEIS A376640}).

In particular, up to relabeling, symmetry, and triviality, there are exactly $4694$ laws of total order at most $4$.  A list can be found \href{https://github.com/teorth/equational_theories/blob/main/data/equations.txt}{here}.  A script for generating them may be found \href{https://github.com/teorth/equational_theories/blob/main/scripts/generate_eqs_list.py}{here}.  The list is sorted first by the total number of operations, then by the number of operations on the LHS. Within each such class we define an order on expressions by lexical order on variables (ordered $x, y, z, w, u, v$).  The equation are arranged to be minimal with respect to this sorting order, thus the LHS will be shorter than or equal than the RHS, and earlier in the lexical order if the LHS and RHS are of equal length.

\input{foundations}
\input{project}
\input{constructions}

\section{Syntactic arguments}\label{syntactic-sec}

Many proofs or refutations of implications (or equivalences) between two equational laws $E,E'$ can be obtained from the syntactic form of the equation.  We discuss some techniques here that were useful in the ETP.

\subsection{Simple rewrites}\label{rewrite-sec}

Many equational laws $E'$ can be formally deduced from a given law $E$ by applying the \emph{Lean} \texttt{rw} tactic to rewrite $E'$ repeatedly by some forward or backward application of $E$ applied to arguments that match some portion of $E$.  For instance, the commutative law \eqref{eq43} clearly implies $\x \op (\y \op \z) \formaleq (\y \op \z) \op \x$ \eqref{eq4531}
by a single such rewrite.  A brute force application of such rewrite methods is already able to directly generate about $\num{15000}$ such implications, including many equivalences to the singleton law \eqref{eq2} and the constant law \eqref{eq46}.  After applying transitive closure, this generates about four million further such implications.

A simple observation that already generates a reasonable number of equivalences is that any equation of the form $\x \formaleq f(\y,\z,\dots)$ necessarily is equivalent to the trivial law $\x \formaleq \y$, by transitivity; similarly, an equation of the form $f(\x,\y) \formaleq g(\z,\w,\dots)$ implies $f(\x,\y) \formaleq f(\x',\y')$; and so forth.  Equivalences of this form were useful early in the project by cutting down the number of distinct equivalence classes of laws that needed to be studied.

\subsection{Matching invariants}

Fix an alphabet $X$. A \emph{matching invariant} is an assignment $I \colon \Magma_X \to {\mathcal I}$ of an object $I(w) \in {\mathcal I}$ in some space ${\mathcal I}$ to each word $w \in \Magma_X$ with the property that if an equational law $w_1 \formaleq w_2$ has matching invariants $I(w_1)=I(w_2)$, then the same matching $I(w'_1) = I(w'_2)$ holds for any consequence $w'_1 \formaleq w'_2$.  In particular, if one law $I(w_1)=I(w_2)$ and $I(w'_1) \neq I(w'_2)$, then the law $w_1 \formaleq w_2$ does not imply the law $w'_1 \formaleq w'_2$.

A simple example of a matching invariant is the multiplicity $(n_x)_{x \in X}$ of variables of a word: if $w_1,w_2$ have all variables $x$ appear the same number of times $n_x$ in both words, then any rewriting of a word $w$ using the law $w_1 \formaleq w_2$ will preserve this property.  Hence, if $w'_1, w'_2$ do not have that each variable appear the same number of times in both words, then $w_1 \formaleq w_2$ cannot imply $w'_1 \formaleq w'_2$.  For instance, the commutative law \eqref{eq43} cannot imply the left-absorptive law \eqref{eq4}.

One source of matching invariants comes from the free magma $\Magma_{X,\Gamma}$ of a theory:

\begin{proposition}[Free magmas and matching invariants]\label{free-inv}  Let $\iota_{X,\Gamma} \colon X \to \Magma_{X,\Gamma}$ be the map associated to the free magma $\Magma_{X,\Gamma}$ for a theory $\Gamma$.  Then the map $I \colon \Magma_X \to \Magma_{X,\Gamma}$ defined by $I(w) \coloneqq \varphi_{\iota_{X,\Gamma}}(w)$ is an invariant.
\end{proposition}

\begin{proof}  Suppose that $w_1 \formaleq w_2$ entails $w'_1 \formaleq w'_2$, and that $I(w_1) = I(w_2)$.  For any $f \colon X \to \Magma_{X,\Gamma}$, the two maps $\varphi_f, \varphi_{f,\Gamma} \circ \varphi_{\iota_{X,\Gamma}} \colon \Magma_X \to \Magma_{X,\Gamma}$ are both homomorphisms that extend $f$, hence agree by the universal property of $\Magma_X$, as displayed by the following commutative diagram:
\[\begin{tikzcd}
	&& X \\
	\\
	{\Magma_X} && {\Magma_{X,\Gamma}} && {\Magma_{X,\Gamma}}
	\arrow[hook, from=1-3, to=3-1]
	\arrow["{\iota_{X,\Gamma}}", pos=0.7, left, from=1-3, to=3-3]
	\arrow["f", pos=0.7, above, from=1-3, to=3-5]
	\arrow["{I = \varphi_{\iota_{X,\Gamma}}}", pos=0.6, above, from=3-1, to=3-3]
	\arrow["{\varphi_f}"', curve={height=18pt}, pos=0.6, below, from=3-1, to=3-5]
	\arrow["{\varphi_{f,\Gamma}}", pos=0.5, above, from=3-3, to=3-5]
\end{tikzcd}\]
In particular, the hypothesis $I(w_1)=I(w_2)$ implies that $\varphi_f(w_1) = \varphi_f(w_2)$ for all $f \colon X \to \Magma_{X,\Gamma}$; that is to say, the magma $\Magma_{X,\Gamma}$ obeys the law $w_1 \formaleq w_2$, and hence also $w'_1 \formaleq w'_2$ by hypothesis.  Thus $\varphi_{\iota_{X,\Gamma}}(w'_1) = \varphi_{\iota_{X,\Gamma}}(w'_2)$, which gives $I(w'_1) = I(w'_2)$ as required.
\end{proof}

\begin{example}  If we take $\Gamma = \{\Eq{4}\}$ to be the theory of the left-absorptive law \eqref{eq4} as described in \Cref{left-absorb}, then the matching invariant $I(w)$ produced by \Cref{free-inv} is the left-most letter of the alphabet $X$ appearing in the word; for instance $I((\x \op \y) \op \z) = \x$.  Thus, for example, the left-absorptive law \eqref{eq4} cannot imply the right-absorptive law \eqref{eq5}.
\end{example}

\begin{example}  If we take $\Gamma = \{\Eq{43}, \Eq{4512}\}$ to be the theory of the commutative law \eqref{eq43} and the associative law \eqref{eq4512}, then by \Cref{semi-group}, the associated invariant $I(w) = \sum_{x \in X} n_x e_x$ is the formal sum of all the generators $e_x$ appearing in the word $w$, in the free abelian semigroup generated by those generators.  This recovers the preceding observation that the multiplicities $(n_x)_{x \in X}$ form a matching invariant.
\end{example}

\begin{example}  Let $n \geq 1$ be a positive integer, and consider the theory $\Gamma = \{\Eq{43}, \Eq{4512}, E_n\}$ consisting of the previous theory $\{\Eq{43}, \Eq{4512}\}$ together with the order-$n$ law $L_x^y x = y$.  One can check that the free magma $\Magma_{X,\Gamma}$ can be described as the free group of exponent $n$ with generators $e_x, x \in X$, with associated map $\iota_{X,\Gamma} \colon x \mapsto e_x$.  The associated matching invariant $I(w) = \sum_{x \in X} n_x e_x$ is essentially the multiplicities $(n_x \hbox{ mod } n)_{x \in X}$ modulo $n$, which gives a slightly stronger criterion than the preceding matching invariant for refuting implications.  For example, the cubic idempotent law $\x \formaleq (\x \op x) \op \x$ \eqref{eq23}
has matching invariants $e_x = 3e_x$ in the $n=2$ case, and hence does not imply the idempotent law $\x \formaleq \x \op \x$ \eqref{eq3} since $e_x \neq 2e_x$ in the $n=2$ case.
\end{example}

In practice, we found that these invariants could be used to establish a significant fraction of the non-implications in the implication graph, although in most cases these non-implications could also be established by other means, for instance through consideration of small finite counterexamples.

\begin{remark}  One can also obtain matching invariants from the free objects associated to theories that involve additional operations beyond the magma operation $\op$, such as an identity element or an inverse operation.  We leave the precise generalization of \Cref{free-inv} to such theories to the interested reader.
\end{remark}

\subsection{Canonization}\label{canon-sec}

One possible way of obtaining refutations of a given implication $E_1 \Rightarrow E_2$ between equational laws is by building a special kind of syntactic model, via certain involutions on elements of the free magma $\Magma_X$ we call \emph{canonizers}.

\begin{definition}
  A function $C:\Magma_X\rightarrow\Magma_X$ is a \emph{canonizer} for an equation $E$ if
  \begin{enumerate}
    \item $C$ is computable.
    \item if $w_1\simeq_E w_2$, then $C(w_1) = C(w_2)$.
  \end{enumerate}
\end{definition}

In fact, a canonizer is simply a matching invariant with target in $\Magma_X$.

We describe a concrete strategy for building such $C$s, which will require a number of definitions.

Let $R$ be an arbitrary function $\Magma_X\rightarrow \Magma_X$.

\begin{definition}\label{def:canon}
  We say that $R$ is (weakly) \emph{collapsing} if for every word $w$, $R(w)$ is a sub-word of $w$ (in the obvious sense).

  A function $\theta : X\rightarrow\Magma_X$ is called a \emph{substitution}, and we can extend $\theta$ to be a homomorphism in the obvious way. We write $w\theta$ instead of $\theta(w)$ for application of substitutions.

  If $E$ is the equation $l\simeq r$, and $l$ is not a variable, we say that $R$ is a \emph{weak canonizer} if for any substitution $\theta$, $R(l\theta)=r\theta$.

  Finally we say that $R$ is \emph{non-overlapping} for $E$ if for every word $w\in\Magma_X$ which is a strict sub-word of $l$ that is not in $X$, and any substitution $\theta$, $R(w\theta) = w\theta$.
\end{definition}

We can then define $C_R : \Magma_X\rightarrow\Magma_X$ as follows:

\begin{align}
  C_R(x) &= x\\
  C_R(w\op w') &= R(C_R(w)\op C_R(w'))
\end{align}

And the following theorem holds.
\begin{theorem}\label{thm:canon}
  Whenever $R$ satisfies all the conditions of Definition \ref{def:canon} then $C_R$ is a canonizer.
\end{theorem}
\begin{proof}
  Assume $w\simeq_{E}w'$. We proceed by induction over the proof of equality.

  The only non trivial case is $w = l\theta$ and $w'=r\theta$ for some substitution $\theta$.
  Then we have
  \begin{align}
    C_R(l\theta) &= R(l(C_R\circ\theta)) \\
                 &= r(C_R\circ\theta)\\
                 &= C_R(r\theta)
  \end{align}
  where the first and third equalities follow from non-overlapping and weak collapsing, and the second from weak canonicity.
\end{proof}

We work through an example to show why this is a useful theorem. Take $E$ to be the equation
\[
  y \op (x \op (y \op (y \op y)))\simeq x
\]
we can take $R$ to be the transformation which sends a term of the form $w \op (v \op (w \op (w \op w)))$ to $v$ for any two words $v, w$, and leaves all other words unchanged.
It is then somewhat easy to show that this transformation satisfies the conditions of theorem \ref{thm:canon}, and so we have a canonizer $C_R$.
This can be used, e.g. to refute the implication of $x = (x \op x) \op (x \op (x \op x))$ from the above equality.

As a conclusion for this section, we note that a very general strategy for building canonizers comes from the theory of \emph{rewrite systems}, see e.g. Baader \& Nipkow \cite{term-rewriting}.
In that setting one defines rewriting as a transformation on words (or terms), and if this transformation is \emph{terminating} and \emph{confluent} (intuitively, rewrites cannot go on forever, or diverge forever), then one may simply pick the transformation which sends a term to its normal form as a canonizer.

Though we note that the non-overlapping criterion seems very similar to the notion of orthogonality in rewriting, we leave the investigation of the precise relationship of the classical theory with the above technique as future work.

\subsection{Unique factorization}

In general, the free magma $\Magma_{X,E}$ for a given equational law $E$, which we can canonically define as $\Magma_X / \sim_E$, is hard to describe explicitly; indeed, from the undecidability of implications between equational laws, such a magma cannot be computably described for arbitrary $E$.  Nevertheless, for some laws it is possible to obtain some partial understanding of $\Magma_{X,E}$ from a syntactic perspective.  For instance, if we can refute the equivalence $w'_1 \sim_E w'_2$ by constructing a counterexample magma $M$ that obeys $E$ but not $w'_1 \formaleq w'_2$, then this implies that the representatives $\iota_{X,E}(w'_1), \iota_{X,E}(w'_2)$ of  $w'_1, w'_2$ in $\Magma_{X,E}$ are distinct.

We illustrate this approach with equations $E$ of the left-absorptive form
\begin{equation}\label{left-absorptive}
\x \formaleq \x \op f(\x,\y,\z)
\end{equation}
for some word $f(\x,\y,\z)$, that are also known to imply the right-idempotent law \eqref{eq378}.  An illustrative example is the law \eqref{eq854} depicted in Figure \ref{fig:854}. Other examples are listed in \Cref{fig:854-like}.

\begin{figure}
  \centering
  \includegraphics[width=0.85\textwidth]{854-like.png}
  \caption{Equations similar to \eqref{eq854} that are of the form \Cref{left-absorptive} (possibly involving a fourth indeterminate $\w$) and imply \eqref{eq378}.  For brevity, $70$ equations equivalent to \eqref{eq4} have been omitted.}
  \label{fig:854-like}
  \end{figure}


\begin{lemma}\label{854} Equation \eqref{eq854} is of the form \eqref{left-absorptive} and implies \eqref{eq378}.
\end{lemma}

\begin{proof}  Clearly we have \eqref{left-absorptive} with $f(\x,\y,\z) \coloneqq (\y \op \z) \op (\x \op \z)$.  From \ref{left-absorptive} we have in any magma obeying \eqref{eq854} that
$$x = x \op f(x,x,S^2 x) = x \op S(x \op S^2 x) = x \op S(x \op f(x,x,x)) = x \op Sx.$$
This implies from a further application of \Cref{left-absorptive} that
$$ y = y \op f(y,x,y) = (y \op Sy) \op ((x \op y) \op Sy) = f(x \op y, y, Sy)$$
and hence by \Cref{left-absorptive} again
$$ (x \op y) \op y = x \op y$$
giving \eqref{eq378}.
\end{proof}

Let $E$ be a law of the form \eqref{left-absorptive} that implies \eqref{eq378}. We define a directed graph $\to_E$ on words in $\Magma_X$ by declaring $w' \to_E w$ if $w \sim_E w'' \op w'$ for some $w' \in \Magma_X$.  By \eqref{eq378} (applied to the quotient magma $\Magma_{X,E} = \Magma_X/\sim_E$), this is equivalent to requiring that $w \sim_E w \op w'$. In particular, from \eqref{left-absorptive} we have $f(x,y,z) \to x$ for all $x,y,z$.  Furthermore, the relation $\to_E$ factors through $\sim_E$: if $w \sim_E \tilde w$ and $w' \sim_E \tilde w'$, then $w' \to_E w$ if and only if $\tilde w \to_E \tilde w$.

Call a word $w \in M_X$ \emph{irreducible} if it is not of the form $w = w_1 \op w_2$ with $w_2 \to_E w_1$.  We can partially understand the equivalence relation $\sim_E$ on irreducible words:

\begin{theorem}[Description of equivalence]\label{irred-desc}  Let $E$ be an equation of the form \eqref{left-absorptive}.  Let $w$ be an irreducible word, and let $w'$ be a word with $w \sim_E w'$.
  \begin{itemize}
    \item[(i)] If $w$ is a product $w = w_1 \op w_2$, then $w'$ takes the form
$$ w' = (((w'_1 \op w'_2) \op v_1) \op \dots \op v_n)$$
for some $w'_1 \sim_E w_1$, $w'_2 \sim_E w_2$, some $n \geq 0$, and some words $v_1, \dots, v_n$ such that for all $0 \leq i < n$, $v_{i+1}$ is of the form
$$ v_{i+1} \sim_E f(x_i,y_i,z_i)$$
for some $x_i, y_i, z_i$ with
$$ x_i \sim_E (((w'_1 \op w'_2) \op v_1) \op \dots \op v_i).$$
In particular, $v_{i+1} \to_E x_i$.
  \item[(ii)] Similarly, if $w \in X$ is a generator of $M_X$, then $w'$ takes the form
$$ w' = ((w \op v_1) \op \dots \op v_n)$$
for some $n \geq 0$, and some words $v_1, \dots, v_n$ such that for all $0 \leq i < n$, $v_{i+1}$ is of the form
$$ v_{i+1} \sim_E f(x_i,y_i,z_i)$$
for some $x_i, y_i, z_i$ with
$$ x_i \sim_E ((w \op v_1) \op \dots \op v_i).$$
In particular, $v_{i+1} \to_E x_i$.
\end{itemize}
Conversely, any word of the above forms is equivalent to $w$.
\end{theorem}

\begin{proof}  We just verify claim (i), as claim (ii) is similar.  The converse direction is clear from \eqref{left-absorptive} (after quotienting by $\sim_E$), so it suffices to prove the forward claim. By the Birkhoff completeness theorem, it suffices to prove that the class of words described by (i) is preserved by any term rewriting operation, in which a term in the word is replaced by an equivalent term using \eqref{left-absorptive}.  Clearly the term being rewritten is in $w'_1$ or $w'_2$ then the form of the word is preserved, and similarly if the term being rewritten is in one of the $v_i$.  The only remaining case is if we are rewriting a term of the form
$$ x_i = (((w'_1 \op w'_2) \op v_1) \op \dots \op v_i).$$
If $i>0$ we can rewrite this term down to $x_{i-1}$, and this still preserves the required form (decrementing $n$ by one).  If $i=0$ then we cannot perform such a rewriting because of the irreducibility of $w_1 \op w_2$ and hence $w'_1 \op w'_2$.  Finally, we can rewrite $x_i$ to $x_i \op v$ where $v$ is of the form
$$ v_i = f(x_i,y,z),$$
and after some relabeling we are again of the required form (now incrementing $n$ by one). This covers all possible term rewriting operations, giving the claim.
\end{proof}

Specializing to the case where $w,w'$ are both irreducible, we conclude

\begin{corollary}[Unique factorization]\label{unique-factorization}  Two irreducible words $w, w'$ are equivalent if and only if they are either the same generator of $X$, or are of the form $w = w_1 \op w_2$, $w' = w'_1 \op w'_2$ with $w_1 \sim_E w'_1$ and $w_2 \sim_E w'_2$.
\end{corollary}

As an application of this corollary, we establish

\begin{proposition}[$\Eq{854}$ does not imply $\Eq{3316}$]\label{854-3316} Equation \eqref{eq854} does not imply \eqref{eq3316}.
\end{proposition}

\begin{proof}(Sketch)
  We work in the free group $\Magma_X$ on two generators $X = \{\x,\y\}$.  It suffices to show that
$$  \x \op \y \not \sim_{E854} \x \op (\y \op (\x \op \y)).$$
Suppose this were not the case, then by \Cref{unique-factorization} one of the following statements must hold:
\begin{itemize}
\item[(i)] $y \to_{E854} x$.
\item[(ii)] $(y \op (x \op y)) \to_{E854} x$.
\item[(iii)] $y \op (x \op y) \sim_{E854} y$.
\end{itemize}
If (i) holds, then we have $x \op y = x$ must hold in $\Magma_X/\sim_E$, hence \eqref{eq854} would imply \eqref{eq4}.  However, it is possible to refute this implication by a finite counterexample.

Similarly, if (iii) held, then \eqref{eq854} would have to imply \eqref{eq10}, but this can also be refuted by a finite magma.

Finally, if (ii) held, then the claim
$$  x \op y \sim x \op (y \op (x \op y))$$
to refute simplifies to
$$  x \op y \sim x$$
and we are back to (i), which we already know not to be the case.
\end{proof}

\input{automated}
\input{austin}
% \input{order5}
\input{higman}
\input{ml}
\input{GUI}
\input{data}
\input{conclusions}

\appendix

\input{numbering}
\section{Author Contributions}

In a \href{https://github.com/teorth/equational_theories/blob/main/paper/contributions.md}{companion document} to this paper, the contributions of each author of this paper to the ETP are described, following the standard CRediT categories\footnote{\url{https://credit.niso.org/}}.  Below are the affiliations and grant acknowledgments of individual participants.  \TODO{Complete the affiliations here.  For contributors with no affiliations to report, enter `Unaffiliated'.}


\begin{itemize}
    \item Matthew Bolan: University of Toronto, matthew.bolan@mail.utoronto.ca
    \item Joachim Breitner: ...
    \item Jose Brox: ...
    \item Mario Carneiro: ...
    \item Martin Dvorak: Institute of Science and Technology Austria, martin.dvorak@matfyz.cz
    \item Andr\'es Goens: University of Amsterdam, a.goens@uva.nl
    \item Aaron Hill: ...
    \item Harald Husum: harald.husum@gmail.com
    \item Zoltan A. Kocsis: University of New South Wales, z.kocsis@unsw.edu.au
    \item Bruno Le Floch: CNRS and Laboratoire de Physique Th\'eorique et Hautes \'Energies, Sorbonne Universit\'e, blefloch@lpthe.jussieu.fr
    \item Lorenzo Luccioli: University of Bologna, lorenzo.luccioli2@unibo.it
    \item Douglas McNeil: dsm054@gmail.com
    \item Alex Meiburg: ...
    \item Pietro Monticone: University of Trento, pietro.monticone@studenti.unitn.it
    \item Pace Nielsen: Department of Mathematics, Brigham Young University, pace@math.byu.edu
    \item Giovanni Paolini: University of Bologna, g.paolini@unibo.it
    \item Marco Petracci: University of Bologna, marco.petracci@studio.unibo.it
    \item Bernhard Reinke: ...
    \item David Renshaw: ...
    \item Marcus Rossel: Barkhausen Institut, marcus.rossel@barkhauseninstitut.org
    \item Cody Roux: Amazon Web Services, cody.roux@gmail.com
    \item J\'er\'emy Scanvic, Laboratoire de Physique, École Normale Supérieure de Lyon, jeremy.scanvic@ens-lyon.fr
    \item Shreyas Srinivas: ...
    \item Anand Rao Tadipatri: ...
    \item Terence Tao: Department of Mathematics, UCLA, tao@math.ucla.edu
    \item Vlad Tsyrklevich: vlad@tsyrklevi.ch
    \item Daniel Weber: ...
    \item Fan Zheng: ...

\end{itemize}


\bibliographystyle{plain}
\bibliography{references}

%delete this later
\listoftodos{}
\end{document}
